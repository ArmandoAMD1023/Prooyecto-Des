\chapter*{INTRODUCCIÓN}
El presente trabajo es un reporte del proyecto final en la asignatura de Desarrollo de Aplicaciones Empresariales del Tecnológico Nacional de México campus León.

En este reporte se presenta una solución a la problemática del Instituto Tecnológico de León que se encuentra organizando un Congreso en Tecnologías de la Información, y por tal motivo necesita una aplicación Web que permita realizar el registro de los participantes, y su asistencia a las conferencias.  

Se pide que la aplicación que se va a desarrollar, sea una aplicación Web moderna, fácil de actualizar, compuesta por un Back-End desarrollado en Asp .Net Core y un Front-End. El framework de desarrollo a usar es libre y además se deberá considerar el diseño y estilo más conveniente, definiendo esquema de colores, tipografías, fondos de pantalla etc. 

Según las peticiones del cliente se poden utilizar frameworks como Bootstrap o algún otro para el Front-End.

En este proyecto se presentan primeramente los aspectos relacionados con el desarrollo de la arquitectura de la información.

Después se exponen los requerimientos de proyecto clasificándolos en requerimientos funcionales y requerimientos no funcionales.

Enseguida se muestra en el trabajo, el desarrollo de los diagramas UML con sus correspondientes imágenes.

Para el almacenamiento de la información se presenta un sistema BD relacional mostrándose sus correspondientes diagramas Entidad Relación y diagrama E-R relacional.

Finalmente se registran las conclusiones sacadas del desarrollo de este proyecto y las referencias de las fuentes de información consultadas.

