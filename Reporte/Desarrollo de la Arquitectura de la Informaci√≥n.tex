\chapter{Desarrollo de la Arquitectura de la Información}

\section{Definición de Objetivos del proyecto}
\subsection{Objetivo General}

Crear una aplicación web moderna, fácil de actualizar, compuesta por un Back-End desarrollado en Asp .Net Core y un Front-End, que permita realizar el registro de los participantes y su asistencia a las conferencias.

\subsection{Objetivos Específicos}
 
\begin{itemize}
	\item Desarrollar la arquitectura de la información.
	\item Elaborar una lista de los requerimientos y clasificarlos.
	\item Desarrollar los diagramas UML.
	\item Elaborar el diagrama E-R y Relacional de la base de datos.
\end{itemize}

\section{Definición de Audiencia}
La presente aplicación va dirigida al personal docente del ITL.

\section{Definición de Contenidos del Proyecto}


\section{Definición de la Arquitectura del Proyecto}

En este proyecto se utilizó la arquitectura modelo-vista-controlador ya que una de sus características es dividir la aplicación en tres partes interactivas entre sí.

\textit{Modelo}: Abarca la funcionalidad y los datos.

\textit{Vista}: Presenta la información al usuario. Puede haber varias vistas para una misma aplicación.

\textit{Controlador}: Administra la entrada del usuario.

Este tipo de arquitectura es muy utilizada para el desarrollo de aplicaciones web y también nos pareció la más adecuada para nuestro proyecto.



\section{Definición de los Sistemas de Navegación}

La razón para diseñar correctamente un sistema de navegación (SN) radica en prevenir que los usuarios puedan hallarse perdidos frente a nuestro web y experimenten sensaciones de confusión.

Los sistemas de navegación son una pieza de vital importancia para la experiencia del usuario de nuestro sitio web. Su aparente sencillez es lo que hace que a menudo olvidemos su importancia y descuidemos su elaboración.

Características de un buen Sistema de Navegación
Todo buen sistema de navegación debe satisfacer al menos los siguientes objetivos:

– Establecer un modo de ir de un sitio a otro dentro de la web. La navegación debe ser clara, concisa, consistente y facilmente identificable dentro de la página. No se puede olvidar que también debe ser transparente: nadie debe ser consciente de que hay un sistema de navegación o de que está usándolo.

– Comunicar al usuario la relación entre el contenido que está visualizando y la navegación del sitio. Debemos permitir que el usuario sepa en todo momento dónde se encuentra, hacia donde puede ir desde este punto y que partes del sitio ha visitado ya.

– Reflejar la arquitectura del sitio que subyace al sistema de navegación. Debemos diferenciar la navegación global, la navegación local o subsecciones y la navegación contextual.

– Permitir volver a la página de inicio rapidamente. En un sitio web la página de inicio sirve como punto de partida y como lugar al que volver cuando nos encontramos perdidos, por eso debe ser sencillo ir a este punto desde cualquier parte del sitio.

– Debe tener un corsé para el crecimiento del site, lo cual significa tener muy claro desde el principio las posibilidades de agrandamiento de ese site y crear un sistema de navegación que permita cubrir esas posibilidades sin desvirtuarse.

Elementos de los Sistemas de Navegación
Barras de Menús
Los menús son la parte más importante de los Sistemas de Navegación. Gracias a ellos, es que el usuario puede navegar libremente por la página, ir a cualquier otra página interna, y recorrer el site sin temor a que su ruta desaparezca. Un menú siempre debe permanecer constante, y lo más recomendable es que no cambie su ubicación ni su diseño en la página (color, tamaño, tipo de letra, etc..)

Hay dos tipos de barras de menús, los menús horizontales y los verticales. Ambos, en mi perspectiva, son igualmente usados en los diseños web. Los dos permiten desarrollarse en menus desplegables (aunque es más recomendable utilizar los menús horizontales para ello), y los dos tambien, permiten desarrollarse en botones de rollover.

Si se desea lograr un buen diseño, es recomendable que la fuente que se utilize sea clara y grande, con un color que contraste con el fondo, para permitir una buena lectura. También es recomendable utilizar [WWW]Sistemas de Etiquetado en el menú, es decir, utilizar íconos en vez de palabras, para que la página adquiera identidad, y sea más fácil para el usuario reconocer puntos estratégicos del site.

Otro punto importante es indicarle al usuario en que parte del menú está. Eso se logra con un cambio en el color del texto, o con una señal (si es una imagen) que indique que el usuario se encuentre en esa página (un blur distinto a la imagen, una flecha, un cambio en el ícono, etc…)
\section{Definición del Diseño Visual del proyecto}
%GUIs
