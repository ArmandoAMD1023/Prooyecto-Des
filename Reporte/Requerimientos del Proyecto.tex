\chapter{Requerimientos del Proyecto}

\newcounter{cRequerimientos}
\setcounter{cRequerimientos}{0}
\newcommand{\IncReq}{\addtocounter{cRequerimientos}{1}}


\section{Funcionales}
\begin{list}{RF\_\IncReq\thecRequerimientos}{}
	\item Pantalla de registro de participantes que contenga un formulario para poder registrar un nuevo participante, guardando los siguientes datos: Nombre, Apellidos, Email, Usuario en Twitter, Avatar y una casilla para aceptar términos y condiciones.
	
	Al hacer clic en Guardar deberá de redirigirnos de nuevo al listado de participantes.
	
	\item Edición  de los datos de un participante.  Al hacer clic sobre el avatar la aplicación deberá de redirigir 
	a una pantalla donde se podrán editar los datos de un participante
	
	\item Listado de Conferencias. Esta página deberá de mostrar una tabla con cada una de las conferencias magistrales, de cada conferencia se deberá de mostrar: horario, título de la conferencia, nombre del conferencista y un enlace al registro de asistencia.
	
	(La lista de conferencias debe ser una lista ya predefinida y no es necesario implementar todo el CRUD, pero la información si debe de venir de la Base de datos).
	
	\item Registro de asistencia a conferencia. Al hacer clic en el enlace de registro, se deberá de mostrar un formulario donde se muestre la conferencia elegida, una lista con los participantes, un check box para confirmar asistencia y un botón para guardar el registro.
	
	
\end{list}

\section{No Funcionales}

\begin{list}{RNF\_\IncReq\thecRequerimientos}{}
	\item En la pantalla de inicio se presentará una landing page con:
	\begin{itemize}
		\item El logotipo de la institución
		\item el nombre y el logotipo del congreso
		\item Un botón de acceso ``Entrar'' que hacer clic en él deberá de redirigir al listado de participantes.
	\end{itemize}
	
	\item En la pantalla de listado de participantes se deberá de mostrar un listado de cada uno de los Participantes que ya están registrados. 
	
	Cada participante se deberá mostrar en una tarjeta en la que aparecerá su nombre completo, enlace a su página de Twitter, ocupación y una imagen tipo avatar.
	(Se deberán utilizar tarjetas para generar el listado, no utilizar tablas)
	
	\item Listado de asistentes a conferencia. Una vez registra el participante, se deberá redirigir a una página donde se muestre un listado con cada uno de los participantes que se han registrado para asistir a dicha conferencia.
\end{list}