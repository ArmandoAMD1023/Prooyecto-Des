\chapter{Requerimientos del Proyecto}

\newcounter{cRequerimientos}
\setcounter{cRequerimientos}{0}
\newcommand{\IncReq}{\addtocounter{cRequerimientos}{1}}


\section{Funcionales}
\begin{list}{RF\_\IncReq\thecRequerimientos}{}
	\item Pantalla de registro de participantes que contenga un formulario para poder registrar un nuevo participante, guardando los siguientes datos: Nombre, Apellidos, Email, Usuario en Twitter, Avatar y una casilla para aceptar términos y condiciones.
	
	Al hacer clic en Guardar deberá de redirigirnos de nuevo al listado de participantes.
	
	\item Edición  de los datos de un participante.  Al hacer clic sobre el avatar la aplicación deberá de redirigir 
	a una pantalla donde se podrán editar los datos de un participante
\end{list}

\section{No Funcionales}

\begin{list}{RNF\_\IncReq\thecRequerimientos}{}
	\item En la pantalla de inicio se presentará una landing page con:
	\begin{itemize}
		\item El logotipo de la institución
		\item el nombre y el logotipo del congreso
		\item Un botón de acceso ``Entrar'' que hacer clic en él deberá de redirigir al listado de participantes.
	\end{itemize}
	
	\item En la pantalla de listado de participantes se deberá de mostrar un listado de cada uno de los Participantes que ya están registrados. 
	
	Cada participante se deberá mostrar en una tarjeta en la que aparecerá su nombre completo, enlace a su página de Twitter, ocupación y una imagen tipo avatar.
	(Se deberán utilizar tarjetas para generar el listado, no utilizar tablas)
\end{list}